\documentclass[a4paper]{memoir}
\usepackage[utf8]{inputenc}
\usepackage[comandos,teorema]{../../senac/preambulo}
\usepackage{minted}
\usepackage{algpseudocode,algorithm}
\usepackage{hyperref}
\hypersetup{pdfpagelayout=TwoPageLeft,
  bookmarksopen=true,
  colorlinks=true,
  urlcolor=blue,
  linkcolor=black,
  pdftitle={T\'opicos de Programa\c c\~ao},
  pdfauthor={R\'egis S. Santos}
}

% Ambiente C
\newminted{c}{linenos}

\theorembodyfont{\normalfont\upshape}
% \newtheorem{ex}{Exemplo}
\newtheorem{eex}{Exerc\'icio}

% Tamanho do vetor
\newcommand{\nv}{0 \ldots n-1}
\newcommand{\un}{1 \ldots n}

\algrenewcommand\algorithmicend{\textbf{fim}}
\algrenewcommand\algorithmicdo{\textbf{faça}}
\algrenewcommand\algorithmicwhile{\textbf{enquanto}}
\algrenewcommand\algorithmicfor{\textbf{para}}
\algrenewcommand\algorithmicif{\textbf{se}}
\algrenewcommand\algorithmicthen{\textbf{então}}
\algrenewcommand\algorithmicelse{\textbf{senão}}
\algrenewcommand\algorithmicreturn{\textbf{devolve}}
\algrenewcommand\algorithmicfunction{\textbf{função}}
\algrenewcommand\algorithmicrepeat{\textbf{repita}}
\algrenewcommand\algorithmicuntil{\textbf{até que}}

\algrenewtext{EndWhile}{\algorithmicend\ \algorithmicwhile}
\algrenewtext{EndFor}{\algorithmicend\ \algorithmicfor}
\algrenewtext{EndIf}{\algorithmicend\ \algorithmicif}
\algrenewtext{EndFunction}{\algorithmicend\ \algorithmicfunction}

\algnewcommand\algorithmicto{\textbf{até}}
\algrenewtext{For}[3]%
{\algorithmicfor\ #1 $\gets$ #2 \algorithmicto\ #3 \algorithmicdo}


\title{T\'opicos de Programa\c c\~ao}
\author{R\'egis S. Santos}
\date{\the\year}
\makeindex

\begin{document}

%suprime numeracao de rodape na capa
\begin{titlingpage}
  \maketitle
\end{titlingpage}

\tableofcontents

\chapter{Introdução}

Livros recomendados: \cite[Cormen]{cormen2001introduction}, \cite[Varizani]{dasgupta2006algorithms}, \cite[Feofiloff]{feofiloff2009algoritmos}.

\section{Lembrando algumas estruturas importantes}

\subsubsection{Um tipo de variável diferente}

\verb|unsigned int| $= \mathbb N$

\subsection{Registro em C}
\index{Registro em C}

Para criar novos tipos de variáveis.

\begin{ex}
Um registro \verb|x| com três campos inteiros.

\begin{minted}{c}
struct{
  int dia, mes, ano;
} x;
\end{minted}

\end{ex}

\begin{ex}
Ponto no plano cartesiano.

\begin{minted}{c}
struct{
  int x,y;
} p1;
\end{minted}

\end{ex}

\begin{ex}
Dando nome ao registro.

\begin{minted}{c}
struct ponto{
  int x,y;
};
struct ponto p1;
struct ponto p2;
\end{minted}

\end{ex}

\begin{ex}
Acessando os campos do registro.

\begin{minted}{c}
p1.x = 3;
p1.y = 2;
\end{minted}

\end{ex}

\section{Ponteiro}
\index{Ponteiro}

Endereço: \&

\begin{ex}
\verb|x| armazenando o endereço de \verb|y|.

\begin{minted}{c}
int *x;
int y = 40;
/* x armazena o endereco de y*/
x = &y;
printf("%d%d",y,*x);
\end{minted}

\end{ex}

\section{Esqueleto em C}
\index{Esqueleto em C}

\begin{ccode}
#include<stdio.h>
int main(){
  /* declaracao de variaveis */
  /* corpo do programa */
  return 0;
}
\end{ccode}

\section{Vetores}
\index{Vetor}

\textbf{Notação}: Se a lista de números está armazenada nas posições $0,1,\ldots,n-1$, diremos que $v[\nv]$ é um \emph{vetor} de inteiros.

Devemos ter $0 \mei n \mei N$.

Se $n = 0$, então $v$ está vazio.

Se $n = N$, então $v$ está cheio.

\section{Operações com vetores}
\index{Operação com!vetor(es)}

\subsection{Busca}
\index{Operação!Busca}
\index{Busca}

Dado um inteiro $x$ e um vetor de inteiros $v[\nv]$, considere o problema de encontrar um índice $k$ tal que $v[k] = x$.

\begin{ex}
Sejam $x = 987, N = 8$ e $n = 5$.

\tkzfigonly{figvetor01}

Então $k = -1$.

\newpage

\begin{minted}{c}
/* Esta funcao recebe um numero x e um vetor v[0..n-1]
   tal que v[k] = x. Se tal k nao existe, devolve -1. */

int Busca(int x, int v[], int n){
  int k;
  k = n - 1;
  while (k >= 0 && v[k] != x)
    k--;
  return k;
}
\end{minted}

\end{ex}

\subsection{Remoção}
\index{Operação!Remoção}
\index{Remoção}

Consiste em retirar do vetor $v[\nv]$ o elemento que tem índice $k$ e fazer com que o vetor resultante tenha índices $0,1,\ldots,n-2$.

\begin{ex}
Vamos remover o elemento de índice $k = 3$.

\tkzfigonly{figvetor02}

\begin{minted}{c}
/* Remove o elemento de indice k do vetor v[0...n-1]
   e devolve o novo valor de n. A funcao supoe 0 <= k < n.*/

int Remove(int k, int v[], int n){
  int j;
  for (j = k; j < n - 1; j++)
    v[j] = v[j+1];
  return n - 1;
}
\end{minted}

\end{ex}

\subsection{Inserção}
\index{Operação!Inserção}
\index{Inserção}

Consiste em introduzir um novo elemento $y$ entre a posição de índice $k-1$ e $k$ no vetor $v[\nv]$. Isto faz sentido somente quando $1 \mei k \mei n - 1$.

Quando $k = 0$, inserimos no início de $v$.

Quando $k = n$, inserimos no final de $v$.

\newpage

\begin{minted}{c}
/* Insere y entre as posicoes k-1 e k de v[0...n-1]
   e devolve o novo valor de n. A funcao supoe 0 <= k <= n.*/

int Insere(int k, int y, int v[], int n){
  int j;
  for (j = n; j > k; j--)
    v[j] = v[j-1];
  v[k] = y;
  return n + 1;
}
\end{minted}

\begin{ex}
Sejam $k = 2$ e $y = 133$.

\tkzfigonly{figvetor03}
\end{ex}

\subsection{Alocação dinâmica de vetor em C}
\index{Alocação dinâmica de!vetor}

\begin{minted}{c}
int *v;
v = (int*) malloc(100*sizeof(int));
\end{minted}

\section{Matrizes bidimensionais}
\index{Matriz}

São estruturas de dados implementadas como vetores de vetores, com $m$ linhas e $n$ colunas.

\begin{minted}{c}
A[i][j] = 1;
\end{minted}

\subsection{Alocação dinâmica de matriz em C}
\index{Alocação dinâmica de!matriz}

\begin{minted}{c}
int **A;
int i;
A = (int**) malloc(m*sizeof(int*));
for (i = 0; i < m; i++)
  A[i] = (int*) malloc(n*sizeof(int));
\end{minted}

\newpage

\subsection{Busca em matriz}
\index{Operação!Busca}
\index{Busca em matriz}

Dado um inteiro $x$ e uma matriz de inteiros $A$ com $m$ linhas e $n$ colunas, encontre dois índices $kl$ e $kc$ tal que $A[kl][kc] = x$.

\begin{minted}{c}
void BuscaMatriz(int x, int **A, int m, int n, int *kl, int *kc){
  for (*kl = n-1; *kl >= 0; (*kl)--) {
    *kc = n - 1;
    while (*kc >= 0 && A[*kl][*kc] != x)
      (*kc)--;
    if (*kc != -1)
      return;
  }
}
\end{minted}

\section{Exercícios}

\begin{eex}
Criar um registro \emph{IndiceMatriz} com campos inteiros \emph{linha} e \emph{coluna} e reescrever a função \verb|BuscaMatriz| usando tal registro.
\end{eex}

\chapter{Análise Assintótica}
\index{Análise Assintótica}

Análise de algoritmos, olhe para expressões como $n+10$ e $n^2+1$ ignorando valores pequenos para $n$. O que interessa são os valores \emph{grandes} para $n$.

Dizemos que as funções $n^2$, $\frac{3}{4}n^2$, $9n^2$, $1000n^2$, $\frac{1}{100}n^2 + 100n$, crescem com a mesma velocidade e por isso são \emph{equivalentes}, quando $n$ é muito grande.

Na análise assintótica todas as funções mencionadas anteriomente estão na mesma \emph{ordem de grandeza}.

Nos interessa apenas olhar para as funções assintóticamente \emph{não negativas}, isto é, funções $f$ tais que $f(n) \mai 0$ para todo $n$ suficientemente grande.

\section{Ordem O}

\begin{defn}
Dadas funções assintóticamentes não negativas $f$ e $g$, dizemos que \emph{$f$ está na ordem O de $g$} e escrevemos $f = O(g)$, se $f(n) \mei c f(n)$ para algum número positivo $c$ e para todo $n$ suficientemente grande.	Ou seja, existe um número positivo $c$ e um número $n_0$ tal que $f(n) \mei c g(n), \forall n > n_0$.

\end{defn}

\begin{ex}
Se $f(n) \mei 999 g (n), \forall n \mai 1000$ então $f$ está na ordem O de $g$. $f = O(g)$.
\end{ex}

\begin{ex}
Suponha que $f(n) = 3 + \frac{2}{n}$ e que $g(n) = n^0 = 1$
\end{ex}

\begin{sol}
$f(n) = 3 + \frac{2}{n} \mei 3 + 1 = 4$
\end{sol}

\begin{ex}
Suponha que $f(n) = n^3$ e que $g(n) = 200n^2$.
\end{ex}

\begin{sol}
Suponha que $f = O(g)$. Então existe um número positivo $c$ e um número $n_0$ tal que $f(n) \mei c g(n), \forall n > n_0$.

Dividindo por $c$ ambos os lados da desigualdade, temos\\
$\frac{1}{c} f(n) \mei g(n), \forall n > n_0$.

Substituindo $f$ e $g$, temos $\frac{1}{c}n^3 \mei 200n^2, \forall n > n_0$.

\[
\frac{1}{c}n \mei 200, \forall n > n_0 \imp \frac{n}{200} \mei c, \forall n > n_0
\]

Absurdo. Portanto, $f$ não está na ordem $O(g).$
\end{sol}

\section{Ordem Omega}

Quando escrevemos $f = O(g)$ queremos dizer que ``a função $f$ está sempre por baixo, ou tocando a função $g$'', a partir de um determinado ponto.

Quando escrevemos $f = \Omega(g)$ queremos dizer que ``a função $f$ está por cima, ou tocando a função $g$'', a partir de um determinado ponto.

\begin{defn}
Dadas funções assintóticamente não negativas $f$ e $g$, dizemos que \emph{$f$ está na ordem Omega de $g$} $(f = \Omega(g))$, se $f(n) \mai c g(n)$ para algum $c$ positivo e para todo $n$ suficientemente grande. Ou seja, $\exists c > 0$ e $n_0$ tais que $f(n) \mai c g(n), \forall n > n_0$.
\end{defn}

\begin{ex}
Se $f(n) \mai \frac{g(n)}{1000}, \forall n \mai 888$, então $f$ está na ordem Omega de $g$.
\end{ex}

\begin{sol}
Devemos provar que se $f = O(g)$, então $g = \Omega(f)$.

$\frac{1}{c}f(n) \mei g(n), \forall n > n_0$.
\end{sol}

\begin{ex}
Prove que $100 \lg n - 10n + 2n \lg n$ está na ordem de Omega de $n \lg n$.
\end{ex}

\begin{sol}
\[
\scriptstyle
f(n) = 100 \lg n - 10n + n \lg n + n \lg n \mai 100 \lg n - 10n + 10n + n \lg n \mai 1n \lg n, \forall n \mai 1024
\]

\[
\therefore f(n) = \Omega(n \lg n)
\]

\end{sol}

\section{Ordem Theta}

\begin{defn}
Dizemos que $f = \Theta(g)$, se $f = O(g)$ e $f = \Omega(g)$. Isto significa que existem números positivos $c$ e $d$, tais que $cg(n) \mei f(n) \mei dg(n)$, para todo $n$ suficientemente grande.
\end{defn}


\chapter{Recorrência}
\index{Recorrência}

\begin{defn}
Uma \emph{recorrência} é uma expressão escrita em termos dela mesma. 
\end{defn}

Por exemplo, $F(n) = F(n-1) + 3n + 2$. O valor de $F(n)$ depende do valor de $F(n-1)$.

Resolver uma recorrência $F$ é encontrar uma ``fórmula fechada'' para $F$ que não depende de $F$ e que dependa somente dos seus parâmetros.

\begin{ex}
Suponha uma recorrência com valor inicial $F(1) = 1$ e\\
$F(n) = F(n-1) + 3n + 2$.
\end{ex}

\begin{sol}

\begin{center}
\begin{tabular}{|c*{6}{|c}}
\hline
$n$ & $1$ & $2$ & $3$ & $4$ & $5$\\ \hline
$F(n)$ & $1$ & $9$ & $20$ & x & x\\ \hline
\end{tabular}
\end{center}

A fórmula é

\[
F(n) = \frac{3}{2}n^2 + \frac{7}{2}n - 4
\]

Façamos a prova por indução:

Para $n = 1$, temos $F(1) = \frac{3}{2}.1^2 + \frac{7}{2}.1 - 4 = 1$.

Tome $n > 1$ e suponha que a fórmula fechada vale para $n-1$.

\begin{eqnarray*}
  F(n) &=& F(n - 1) + 3n + 2 \hfill \\
   &=& \frac{3}{2}(n - 1)^2  + \frac{7}{2}(n - 1) - 4 + 3n + 2 \hfill \\
   &=& \frac{3}{2}(n^2  - 2n + 1) + \frac{7}{2}(n - 1) - 4 + 3n + 2 \hfill \\
   &=& \frac{{3n^2  - 6n + 3 + 7n - 7 - 8 + 6n + 4}}{2} \hfill \\
   &=& \frac{3}{2}n^2  + \frac{7}{2}n - 4 \hfill \\ 
\end{eqnarray*} 

\newpage 

Vamos mostrar que $F(n) = \frac{3}{2}n^2 + \frac{7}{2}n - 4 = \Theta(n^2)$.

Primeiro vamos mostrar que $F(n) = O(n^2)$.

\[
\begin{gathered}
  F(n) = \frac{3}{2}n^2  + \frac{7}{2}n - 4 \hfill \\
  2F(n) = 3n^2  + 7n - 8 \hfill \\
  2F(n) \mei 3n^2  + 7n,\forall n \mai 0 \hfill \\
  2F(n) \mei 3n^2  + 7n^2  = 10n^2 ,\forall n \mai 0 \hfill \\
  F(n) \mei 5n^2 ,\forall n \mai 0 \hfill \\ 
\end{gathered} 
\]

Logo $n_0 = 0$ e $c = 5$.

Vamos mostrar que $F(n) = \Omega(n^2)$.

\[
F(n) = \frac{3}{2}n^2  + \frac{7}{2}n - 4 \mai \frac{3}{2}n^2 ,\forall n > 2
\]

Logo $n_0 = 2$ e $c = \frac{3}{2}$.

\end{sol}

\begin{ex}
Suponha a recorrência $F(n) = F \left( \frac{n}{2} \right) + 3$ com valor inicial $F(1) = 5$.
\end{ex}

\begin{sol}
\begin{center}
\begin{tabular}{|c*{7}{|c}}
\hline
$n$ & $1$ & $2$ & $3$ & $4$ & $5$ & $6$\\ \hline
$F(n)$ & $5$ & $8$ & ND & $11$ & ND & ND\\ \hline
\end{tabular}
\end{center}

Podemos escrever $n = 2^k, k \in \N$.

\begin{equation*}
  F(n) =
  \begin{cases}
    F(1)=5 \\
    F(2^k) = F(2^{k-1}) + 3
  \end{cases}
\end{equation*}

\begin{center}
\begin{tabular}{|c*{5}{|c}}
\hline
$k$ & $0$ & $1$ & $2$ & $3$\\ \hline
$F(2^k)$ & $5$ & $8$ & $11$ & x\\ \hline
\end{tabular}
\end{center}

\begin{eqnarray*}
  F(2^k) &=& F(2^{k - 1}) + 3 \hfill \\
   &=& F(2^{k - 2}) + 2.3 \hfill \\
   &=& F(2^{k - 3}) + 3.3 \hfill \\
   &=& F(2^{k - 4}) + 4.3 \hfill \\
   &=& F(2^{k - 5}) + 5.3 \hfill \\ 
\end{eqnarray*}

\newpage 

Continuando este processo até o expoente de $2$, dentro de $F$, for igual a $0$, temos:

\begin{eqnarray*}
  F(2^k) &=& F(2^{k - k}) + k.3 \hfill \\
   &=& F(1) + k.3 \hfill \\
   &=& 5 + 3k \hfill \\ 
\end{eqnarray*}

Escrevendo $F$ em função de $n$, temos:

\begin{eqnarray*}
  n &=& 2^k  \hfill \\
  \lg n &=& \lg 2^k  \hfill \\
  \lg n &=& k \hfill \\ 
\end{eqnarray*}

Portanto, $F(n) = 5 + 3 \lg n$. ($\lg = \log _2$)

\

Mostre que $F(n) = 5 + 3 \lg n = \Theta(\lg n)$

Vamos mostrar que $F(n) = O(\lg n)$.

\begin{eqnarray*}
  F(n) &=& 3\lg n + 5 \hfill \\
  &\mei& 3 + \lg n + 5\lg n \hfill \\
   &=& 8\lg n,\forall n > 1 \hfill \\ 
\end{eqnarray*}

Vamos mostrar que $F(n) = \Omega(\lg n)$.

\[
F(n) = 3\lg n + 5 \mei 3 \lg n,\forall n \mai 1
\]

\end{sol}


\chapter{Algoritmos Recursivos}
\index{Recursão}

Algoritmos recursivos são aplicados em problemas com uma \emph{estrutura recursiva}.

\begin{ex}[Soma]\index{Recursão!Soma}
Calcule a soma dos elementos do vetor $A[\un]$ recursivamente, $n \mai 0$.
\end{ex}

\begin{sol}
\footnote{\texttt{http://en.wikibooks.org/wiki/LaTeX/Algorithms\_and\_Pseudocode\#While-loops}}
\begin{algorithm}
\caption*{Soma elementos do vetor}
\begin{algorithmic}[1]
\Function{Soma}{A,n}
\If {$n = 0$}
  \State \Return $0$
\Else
  \State \Return \texttt{Soma(A,n-1) + A[n]}
\EndIf
\EndFunction
\end{algorithmic}
\end{algorithm}

\

Aplicando a recursão no final de $A$. ($A[k \ldots n]$)

\begin{algorithm}
\caption*{Soma recursiva}
\begin{algorithmic}[1]
\Function{Soma2}{A,k,n}
\If {$k > n$}
    \State \Return $0$
\Else
    \State \Return \texttt{Soma2(A,k+1,n) + A[k]}
\EndIf
\EndFunction
\end{algorithmic}
\end{algorithm}

\end{sol}

\newpage 

\begin{ex}[Máximo de um vetor]
Escreva um algoritmo iterativo e um recursivo para o seguinte problema: encontrar o valor de um maior elemento do vetor $A[\un]$.
\end{ex}

\begin{sol}
\textbf{Algoritmo iterativo}
\index{Máximo}

\begin{algorithm}
\caption*{Máximo}
\begin{algorithmic}[1]
\Function{Max}{A,n}
\State $x \gets A[1]$
\For{i}{2}{n}
  \If {$A[i] > x$}
    \State {$x \gets A[i]$}
  \EndIf
  \State {\Return $x$}
\EndFor
\EndFunction
\end{algorithmic}
\end{algorithm}


\textbf{Algoritmo recursivo}
\index{Recursão!Máximo}

\begin{algorithm}
\caption*{Máximo recursivo}
\begin{algorithmic}[1]
\Function{MaxRec}{A,n}
  \If {$n = 1$}
    \State \Return $A[1]$
  \Else
    \State {$x \gets$ \texttt{MaxRec(A,n-1)}}
    \If {$A[n] > x$}
	\State \Return $A[n]$
    \Else
	\State \Return $x$
    \EndIf
  \EndIf
\EndFunction
\end{algorithmic}
\end{algorithm}

\end{sol}

\newpage 

\begin{ex}[Busca simples em vetor]
Escreva um algoritmo iterativo e recursivo para resolver o problema de busca simples no vetor $A[\un]$.
\end{ex}

\begin{sol}
\textbf{Algoritmo iterativo}

\begin{algorithm}
\caption*{Busca iterativa}
\begin{algorithmic}[1]
\Function{Busca}{A,n,x}
  \State {$i \gets 1$}
  \While{$i \mei n$ e $A[i] \ne x$}
    \State $i \gets i + 1$
    \If {$i > n$}
      \State \Return nao
    \Else
      \State \Return sim
    \EndIf
  \EndWhile
\EndFunction
\end{algorithmic}
\end{algorithm}

\textbf{Algoritmo recursivo}
\index{Recursão!Busca}

\begin{algorithm}
\caption*{Busca recursiva}
\begin{algorithmic}[1]
\Function{BuscaRec}{A,n,x}
  \If{$n = 0$}
    \State \Return nao
    \Else
      \If{$A[n] = x$}
	\State \Return sim
      \Else
	\State \Return \texttt{BuscaRec(A,n-1,x)}
      \EndIf
  \EndIf
\EndFunction
\end{algorithmic}
\end{algorithm}

Uma outra forma de busca pode ser dividindo o vetor ao meio. Considere $n = 2^i$.

Considere um vetor onde o 1º elemento é $k$ e o ultimo é $n$.

Note que o tamanho do vetor é $n - k + 1$.

\newpage 

\begin{algorithm}
\caption*{Busca recursiva dividindo o vetor ao meio}
\begin{algorithmic}[1]
\Function{BuscaRec2}{A,k,n,x}
  \If {$n - k + 1 = 0$}
    \Else
      \If {$A[n] = x$}
	\State \Return sim
      \Else
	\State \Return nao
      \EndIf
    \State \Return \texttt{(BuscaRec(A,k,k + $\frac{n-k+1}{2} - 1$,x))}
    \State \texttt{BuscaRec(A,k + $\frac{n-k+1}{2}$,n,x)}
  \EndIf
\EndFunction
\end{algorithmic}
\end{algorithm}

\end{sol}

\begin{ex}[Remoção em um vetor]\index{Recursão!Remoção}
Remoção em um vetor $A[\un]$.
\end{ex}

\begin{sol}

\begin{algorithm}
\caption*{Remoção recursiva}
\begin{algorithmic}[1]
\Function{RemoveRec}{A,k,n}
  \If {$k = n$}
    \State \Return $n-1$
  \Else
    \State $A[k] \gets A[k+1]$
  \EndIf
  \State \Return \texttt{RemoveRec(A,k+1,n)}
\EndFunction
\end{algorithmic}
\end{algorithm}

\end{sol}

\newpage 

\begin{ex}[Inserção]\index{Recursão!Inserção}
Inserção em um vetor $A[\un]$.
\end{ex}

\begin{sol}

\begin{algorithm}
\caption*{Inserção recursiva}
\begin{algorithmic}[1]
\Function{InsereRec}{k,y,A,n}
  \If {$k = n + 1$}
    \State $A[n+1] = y$
  \Else
    \State $A[n+1] = A[n]$
  \EndIf
  \State \texttt{InsereRec(k,y,A,n-1)}
  \State \Return $n + 1$
\EndFunction
\end{algorithmic}
\end{algorithm}

\end{sol}

\section{Exercícios}

\begin{eex}[Fatorial]\index{Recursão!Fatorial}
Escreva um algoritmo recursivo que recebe um valor inteiro $n$ e devolve $n!$.

\begin{equation*}
  n! = 
  \begin{cases}
    \hfill 1 	& \mbox{, se } n = 0 \\
    \hfill n*(n - 1) 	& \mbox{, se } n > 0
  \end{cases}
\end{equation*}

\end{eex}

\begin{sol}

\begin{algorithmic}[1]
\Function{Fatorial}{n}
  \If {$n = 0$}
    \State \Return $1$
  \Else
    \State \Return $n * $ \texttt{Fatorial(n - 1)}
  \EndIf
\EndFunction
\end{algorithmic}

\end{sol}

\newpage 

\begin{eex}[Fibonacci]\index{Recursão!Fibonacci}
Escreva um algoritmo recursivo que recebe um valor inteiro $n$ e devolve \texttt{fib(n)}, onde

\begin{equation*}
  \texttt{fib(n)} =
  \begin{cases}
     \qquad \qquad 0 	& \mbox{, se } n = 0 \\
     \qquad \qquad 1 	& \mbox{, se } n = 1 \\
    \hfill \texttt{fib(n - 1)} + \texttt{fib(n - 2)} 	& \mbox{, se } n > 1
  \end{cases}
\end{equation*}

\end{eex}

\begin{sol}
 
\begin{algorithmic}[1]
\Function{fib}{n}
  \If {$n = 0$}
    \State \Return $0$
  \Else
    \If {$n = 1$}
      \State \Return $1$
    \Else
      \State \Return \texttt{fib(n - 1) + fib(n - 2)}
    \EndIf
  \EndIf
\EndFunction
\end{algorithmic}

\end{sol}

\begin{eex}[Torre de Hanói]\index{Recursão!Torre de Hanói}
O problema consiste de três pinos \emph{Origem, auxiliar, destino} e uma torre com $n$ discos no pino \emph{Origem} para o pino \emph{Destino} usando o pino \emph{Auxiliar} como auxiliar.
\end{eex}

\begin{sol}

\begin{algorithmic}[1]
\Function{moveTorre}{n,origem,destino,auxiliar}
  \If {$n = 1$}
    \State escreva ``mova disco de '', \texttt{origem}, ``para '', \texttt{destino}
  \Else
    \State \texttt{moveTorre(n - 1,origem,auxiliar,destino)}
    \State escreva ``mova disco de '', \texttt{origem}, ``para '', \texttt{destino}
    \State \texttt{moveTorre(n - 1,auxiliar,destino,origem)}
  \EndIf
\EndFunction
\end{algorithmic}

\end{sol}

\newpage 

\section{Ordenação}

\begin{ex}
Rearranjar um vetor de inteiros $a[\un]$ de modo que ele fique em ordem crescente.
\end{ex}


\begin{sol}
\index{Ordenação por!Inserção}

\begin{algorithm}
\caption*{Ordenação por Inserção}
\begin{algorithmic}[1]
\Function{OrdenacaoInser}{A,n}
\For{j}{2}{n} \Comment{$n$}
  \State {$x \gets a[j]$} \Comment{$n - 1$}
  \State {$i \gets j - 1$} \Comment{$n - 1$}
  \While {$i > 0$ e $A[i] > x$} \Comment{$\frac{(n - 1)(n - 2)}{2}$}
    \State $A[i + 1] \gets A[i]$ \Comment{$\frac{n(n - 1)}{2}$}
    \State $i \gets i - 1$ \Comment{$\frac{n(n - 1)}{2}$}
  \EndWhile
  \State $A[i + 1] \gets x$ \Comment{$n - 1$}
\EndFor
\EndFunction
\end{algorithmic}
\end{algorithm}

Calculo do tempo:

\[
\begin{gathered}
  T(n) = n + n - 1 + n - 1 + \frac{{(n - 1)(n + 2)}}{2} + n(n - 1) + n - 1 \hfill \\
  T(n) = \frac{3}{2}n^2  + \frac{7}{2}n - 4 \hfill \\ 
\end{gathered} 
\]

Mostrar que $T(n) = \Theta(n^2)$.
\end{sol}

\newpage 

\section{Ordenação por Inserção Recursiva}
\index{Ordenação por!Inserção Recursiva}

\begin{sol}
 
\begin{algorithm}
\caption*{Ordenação por Inserção Recursiva}
\begin{algorithmic}[1]
\Function{OrdenacaoInserRec}{A,n}
  \If {$n > 1$} \Comment{$1$}
    \State \texttt{OrdenacaoInserRec(A,n-1)} \Comment{$T(n - 1)$}
    \State $x \gets A[n]$  \Comment{$1$}
    \State $i \gets n - 1$  \Comment{$1$}
    \While {$i > 0$ e $A[i] > x$} \Comment{$n$}
      \State $A[i + 1] \gets A[i]$ \Comment{$n - 1$}
      \State $i \gets i - 1$ \Comment{$n - 1$}
    \EndWhile
    \State $A[i + 1] \gets x$ \Comment{$1$}
  \EndIf
\EndFunction
\end{algorithmic}
\end{algorithm}

\[
\left\{ \begin{gathered}
  T(1) = 1 \hfill \\
  T(n) = T(n - 1) + 3n + 2 \hfill \\
\end{gathered} \right.
\]

Temos que $T(n) = an^2 + bn + c$.

\begin{center}
\begin{tabular}{|c*{4}{|c}}
\hline
$n$ & $1$ & $2$ & $3$\\ \hline
$T(n)$ & $1$ & $9$ & $20$\\ \hline
\end{tabular}
\end{center}

\[
\left\{ \begin{gathered}
  T(1) = a + b + c = 1 \hfill \\
  T(2) = 4a + 2b + c = 9 \hfill \\
  T(3) = 9a + 3b + c = 20 \hfill \\ 
\end{gathered}  \right.
\]

Resolvendo o sistema linear, temos que $a = \frac{3}{2}, b = \frac{7}{2}$ e $c = -4$.

Portanto, $T(n) = \frac{3}{2}n^2 + \frac{7}{2}n - 4$.
\end{sol}

\newpage 

\begin{ex}[Ordenação de um vetor de inteiros]\index{Ordenação por!Seleção}
Considere um algoritmo com a seguinte ideia: no início de cada iteração $A[i \ldots j - 1]$ está em ordem crescente e contém os elementos pequenos de $A$; o vetor $A[j \ldots n]$ contém os elementos grandes. Escreva uma versão iterativa do algoritmo.
\end{ex}

\begin{sol}
Dica: suponha um algoritmo que encontre um menor elemento no vetor $A[p \ldots q]$, \texttt{Min(A,p,q)}.

Suponha um algoritmo que troque de posição dois elementos de $A$,\\
\texttt{Troque(A,p,q)}.

\begin{algorithm}
\caption*{Ordenação por Seleção}
\begin{algorithmic}[1]
\Function{OrdSelec}{A,n} \Comment{$T(n)$}
  \State $k \gets$ \texttt{Min(A,1,n)} \Comment{$3(n)$}
  \State \texttt{Troca(A,1,k)} \Comment{$3$}
  \For{j}{2}{n - 1} \Comment{$n - 1$}
    \State $k \gets$  \texttt{Min(A,j,n)} \Comment{$\scriptstyle (n - 2)T'(n - j + 1)(n - 2).3(n - j + 1) \mei (n - 2).3n$}
    \State \texttt{Troca(A,j,k)} \Comment{$(n - 2).3$}
  \EndFor
\EndFunction
\end{algorithmic}
\end{algorithm}

\

\begin{algorithmic}[1]
\Function{Troca}{A,p,q}
  \State $x \gets A[p]$
  \State $A[p] \gets A[q]$
  \State $A[q] \gets x$
\EndFunction
\end{algorithmic}

$T(n) = 3$

\

\begin{algorithmic}[1]
\Function{Min}{A,p,q} \Comment{$T(n')$}
  \State $k \gets p$ \Comment{$1$}
  \For{j}{p + 1}{q} \Comment{$n'$}
    \If {$A[j] < A[k]$} \Comment{$n' - 1$}
      \State $k \gets j$ \Comment{$n' - 1$}
    \EndIf
    \State \Return $k$ \Comment{$1$}
  \EndFor
\EndFunction
\end{algorithmic}

$T(n') = 3n'$

\

Então,

\[
\begin{gathered}
  T(n) \mei 3n + 3 + n - 1 + 3n^2  - 6n + 3n - 6 \hfill \\
  T(n) \mei 3n^2  + n - 4 \hfill \\ 
\end{gathered} 
\]

\end{sol}

\newpage

\begin{ex}[Algoritmo MergeSort]\index{Ordenação por!MergeSort}
Ordenar um vetor de inteiros $A[p \ldots r]$.
\end{ex}

\begin{sol}

\begin{algorithm}
\caption*{MergeSort}
\begin{algorithmic}[1]
\Function{MergeSort}{A,p,r}		\Comment{$T'(n)$}
  \If {$p < r$}				\Comment{$1$}
    \State $q \gets \left\lfloor {\frac{{p + r}}{2}} \right\rfloor$\Comment{$1$}
    \State \texttt{MergeSort(A,p,q)}	
\Comment{$T'(\left\lceil {{\raise0.5ex\hbox{$\scriptstyle n$}
\kern-0.1em/\kern-0.15em
\lower0.25ex\hbox{$\scriptstyle 2$}}} \right\rceil )$}
    \State \texttt{MergeSort(A,q+1,r)}
\Comment{$T'(\left\lfloor {{\raise0.5ex\hbox{$\scriptstyle n$}
\kern-0.1em/\kern-0.15em
\lower0.25ex\hbox{$\scriptstyle 2$}}} \right\rfloor )$}
    \State \texttt{Intercala(A,p,q,r)}	\Comment{$6n + 5$}
  \EndIf
\EndFunction
\end{algorithmic}
\end{algorithm}

\

\begin{algorithmic}[1]
\Function{Intercala}{A,p,q,r}		\Comment{$T(n)$}
  \For{i}{p}{q}				\Comment{$n + 2$}
    \State $B[i] \gets A[i]$		\Comment{$n$}
  \EndFor
  \For{j}{q+1}{r}
    \State $B[r + q + 1 - j] \gets A[j]$
  \EndFor
  \State $i \gets p$			\Comment{$1$}
  \State $j \gets r$			\Comment{$1$}
  \For{k}{p}{r}				\Comment{$n + 1$}
    \If {$B[i] \mei B[j]$}		\Comment{$n$}
      \State $A[k] \gets B[i]$		\Comment{$2n$}
      \State $i \gets i + 1$
    \Else
      \State $A[k] \gets B[j]$
      \State $j \gets j - 1$
    \EndIf
  \EndFor
\EndFunction
\end{algorithmic}

$T(n) = 6n + 5$.

\[
\left\{ \begin{gathered}
  T'(1) = 1 \hfill \\
  T'(n) = T'\left( {\left\lceil {{\raise0.5ex\hbox{$\scriptstyle n$}
\kern-0.1em/\kern-0.15em
\lower0.25ex\hbox{$\scriptstyle 2$}}} \right\rceil } \right) + T'\left( {\left\lfloor {{\raise0.5ex\hbox{$\scriptstyle n$}
\kern-0.1em/\kern-0.15em
\lower0.25ex\hbox{$\scriptstyle 2$}}} \right\rfloor } \right) + 6n + 7 \hfill \\ 
\end{gathered}  \right.
\]

Depois de simplificar $T'$, temos

\[
\begin{gathered}
  T'(n) = 2T'\left( {{\raise0.5ex\hbox{$\scriptstyle n$}
\kern-0.1em/\kern-0.15em
\lower0.25ex\hbox{$\scriptstyle 2$}}} \right) + 6n + 7 \hfill \\
  T'(n) \mei cn\lg n,n \mai n_0  \hfill \\ 
\end{gathered} 
\]

\begin{center}
\begin{tabular}{|c*{4}{|c}}
\hline
$n$ & $1$ & $2$ & $3$\\ \hline
$T'(n)$ & $1$ & $21$ & $47$\\ \hline
\end{tabular}
\end{center}

Temos, $T'(n) \mei 16 n \lg n$
\end{sol}

\begin{ex}[Algoritmo QuickSort]\index{Ordenação por!QuickSort}
Ordenar um vetor de inteiros $A[p \ldots r]$. Usar a estratégia de divisão e conquista.
\end{ex}

\begin{sol}

\begin{algorithm}
\caption*{Separe}
\begin{algorithmic}[1]
\Function{Separe}{A,p,r}		\Comment{$T(n) = 5n + 9$}
  \State $x \gets A[p]$
  \State $i \gets p - 1$
  \State $j \gets r + 1$
  \While {$0 = 0$}
    \Repeat
      \State $j \gets j - 1$
    \Until $A[j] \mei x$
    \Repeat
      \State $i \gets i + 1$
    \Until $A[i] \mei x$
    \If {$i < j$}
      \State \texttt{Troca(A,i,j)}
    \Else
      \State \Return $j$
    \EndIf
  \EndWhile
\EndFunction
\end{algorithmic}
\end{algorithm}

\begin{algorithm}
\caption*{QuickSort}
\begin{algorithmic}[1]
\Function{QuickSort}{A,p,r}
  \If {$p < r$}
    \State $q \gets$ \texttt{Separe(A,p,r)}
    \State \texttt{QuickSort(A,p,q)}
    \State \texttt{QuickSort(A,q+1,r)}
  \EndIf
\EndFunction
\end{algorithmic}
\end{algorithm}

\end{sol}

\chapter*{Exemplos de Pseudocódigo no \LaTeX}

\begin{algorithm}
\caption*{Valor Absoluto}
\begin{algorithmic}[1]
\Function{Absoluto}{x}
\If {$x < 0$}
    \State \Return $-x$
\Else
    \State \Return $x$
\EndIf
\EndFunction
\end{algorithmic}
\end{algorithm}

\begin{algorithm}
\caption*{Exemplo do \texttt{for}}
\begin{algorithmic}[1]
\For{i}{1}{n}
  \State {$A[i] \gets i + 1$} \Comment{Preenche o vetor}
\EndFor
\end{algorithmic}
\end{algorithm}

\begin{algorithm}
\caption*{Exemplo do \texttt{while}}
\begin{algorithmic}[1]
\While {$i \mei n$}
  \State $i \gets i + 1$
\EndWhile
\end{algorithmic}
\end{algorithm}



%***********************************************
\bibliographystyle{abbrv}
\bibliography{../../senac/refs}

\printindex

\end{document}
